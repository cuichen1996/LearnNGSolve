\chapter{Sobolev Spaces}
\label{sec_sobolev}
%
In this section, we introduce the concept of generalized derivatives, we
define families of normed function spaces, and prove inequalities between
them.
Let $\Omega$ be an open subset of ${\mathbb R}^d$, either bounded or unbounded.






\section{Generalized derivatives}

Let $\alpha = (\alpha_1, \ldots , \alpha_d) \in {\mathbb N}_0^d$ be a multi-index,
$| \alpha | = \sum \alpha_i$, 
and define the classical differential operator for functions in $C^\infty (\Omega)$
$$
D^\alpha = 
        \left( \frac{\partial}{\partial x_1} \right)^{\alpha_1} \cdots
        \left( \frac{\partial}{\partial x_n} \right)^{\alpha_d}.
$$
%
For a function $u \in C(\Omega)$, the support is defined as 
$$
\operatorname{supp} \{ u \} := \overline{ \{ x \in \Omega : u(x) \neq 0 \} }.
$$
This is a compact set if and only if it is bounded. We say $u$ has
compact support in $\Omega$, if $\operatorname{supp} u \subset \Omega$.
If $\Omega$ is a bounded domain, then $u$ has compact support in $\Omega$
if and only if $u$ vanishes in a neighbourhood of $\partial \Omega$.

\medskip
\noindent
The space of smooth functions with compact support is denoted as
\begin{equation}
{\cal D} (\Omega) := C_0^\infty (\Omega) := 
\{ u \in C^\infty(\Omega) : \mbox{ $u$ has compact support in $\Omega$} \}.
\end{equation}
%
For a smooth function $u \in C^{|\alpha|}(\Omega)$, there holds the
formula of integration by parts
\begin{equation} \label{equ_intbyparts}
\int_\Omega D^\alpha u \varphi \, dx = (-1)^{|\alpha|} \int_\Omega u D^\alpha \varphi \, dx 
\qquad \forall \, \varphi \in {\cal D}(\Omega).
\end{equation}
The $L_2$ inner product with a function $u$ in $C(\Omega)$ defines the linear
functional on ${\cal D}$
$$
u(\varphi) := \left< u, \varphi \right>_{{\cal D}^\prime \times {\cal D}} := \int_\Omega u \varphi \, dx.
$$
We call these functionals in ${\cal D}^\prime$ distributions.
When $u$ is a function, we identify it with the generated distribution.
The formula (\ref{equ_intbyparts}) is valid for functions $u \in C^\alpha$.
The strong regularity is needed only on the left hand side. Thus, we use
the less demanding right hand side to extend the definition of differentiation
for distributions:

\begin{definition} For $u \in {\cal D}^\prime$, we define 
$g \in {\cal D}^\prime$ to be the generalized 
derivative $D_g^\alpha u$ of $u$ by
$$
\left<g, \varphi \right>_{{\cal D}^\prime \times {\cal D}} = 
(-1)^{|\alpha|} \left<u, D^\alpha \varphi \right>_{{\cal D}^\prime \times {\cal D}}  \qquad \forall \, \varphi \in {\cal D}
$$
\end{definition}
%
\noindent
If $u \in C^\alpha$, then $D_g^\alpha$ coincides with $D^\alpha$. 

\noindent
The function space of {\bf locally integrable} functions on $\Omega$ 
is called
$$
L_1^{loc} (\Omega) = \{ u : u_K \in L_1(K) \; \forall \mbox{ compact } K 
\subset \Omega \}.
$$
It contains functions which can behave very badly near $\partial \Omega$.
E.g., $e^{e^{1/x}}$ is in $L_{loc}^1 (0,1)$. If $\Omega$ is unbounded, then
the constant function $1$ is in $L_1^{loc}$, but not in $L_1$.


\begin{definition} For $u \in L_1^{loc}$, we call $g$ the weak derivative
$D_w^\alpha u$, if $g \in L_1^{loc}$ satisfies
$$
\int_\Omega g(x) \varphi(x) \, dx =
(-1)^{|\alpha|} \int_\Omega u(x) D^\alpha \varphi (x) \, dx  \qquad \forall \, \varphi \in {\cal D}.
$$
\end{definition}
The weak derivative is more general than the classical derivative, but more
restrictive than the generalized derivative.


\begin{example} 
Let $\Omega = (-1,1)$ and 
$$
u(x) = \left\{ 
        \begin{array}{cl}
        1+x & \quad x \leq 0 \\
        1-x & \quad x > 0
        \end{array}
        \right\}
$$
Then, 
$$
g(x) = \left\{ 
        \begin{array}{cl}
        1 & \quad x \leq 0 \\
        -1 & \quad x > 0
        \end{array}
        \right\}
$$
is the first generalized derivative $D^1_g$ of $u$, which is also
a weak derivative. The second generalized derivative $h$ is 
$$
\left< h, \varphi \right> = -2 \varphi(0)  \qquad \forall \, \varphi \in {\cal D}
$$
It is not a weak derivative.
\end{example}

In the following, we will focus on weak derivatives. Unless it is essential
we will skip the sub-scripts $w$ and $g$.











\section{Sobolev spaces}

For $k \in {\mathbb N}_0$ and $1 \leq p < \infty$, we define the Sobolev norms 
$$
\| u \|_{W_p^k(\Omega)} := \left( \sum_{|\alpha| \leq k} \| D^\alpha u \|_{L_p}^p \right)^{1/p},
$$
for $k \in {\mathbb N}_0$ we set
$$
\| u \|_{W_\infty^k(\Omega)} := \max_{|\alpha| \leq k}  \| D^\alpha u \|_{L_\infty}.
$$
In both cases, we define the {\bf Sobolev spaces} via
$$
W_p^k(\Omega) = \{ u \in L_1^{loc} : \| u \|_{W_p^k} < \infty \}
$$

In the previous chapter we have seen the importance of complete spaces.
This is the case for Sobolev spaces:
\begin{theorem} The Sobolev space $W_p^k(\Omega)$ is a Banach space.
\end{theorem}
\noindent
{\em Proof:} Let $v_j$ be a Cauchy sequence with respect to $\| \cdot \|_{W_p^k}$. This implies that $D^\alpha v_j$ is a Cauchy sequence in $L_p$, and 
thus converges to some $v^\alpha$ in $\|. \|_{L_p}$. 

We verify that $D^\alpha v_j \rightarrow v^\alpha$ implies 
$\int_\Omega D^\alpha v_j \varphi \, dx \rightarrow \int_\Omega v^\alpha \varphi \, dx$ for all $\varphi \in {\cal D}$. Let $K$ be the compact support 
of $\varphi$. There holds 
\begin{eqnarray*}
\int_\Omega (D^\alpha v_j - v^\alpha) \varphi \, dx & = & 
\int_{K} (D^\alpha v_j - v^\alpha) \varphi \, dx  \\
& \leq & \| D^\alpha v_j - v^\alpha \|_{L_1(K)} \| \varphi \|_{L_\infty} \\
& \leq & \| D^\alpha v_j - v^\alpha \|_{L_p(K)} \| \varphi \|_{L_\infty}
\rightarrow 0
\end{eqnarray*}
Finally, we have to check that $v^\alpha$ is the weak derivative of $v$:
\begin{eqnarray*}
\int v^\alpha \varphi \, dx & = & 
        \lim_{j\rightarrow \infty} \int_\Omega D^\alpha v_j \varphi \, dx \\
        & = & \lim_{j\rightarrow \infty} (-1)^{|\alpha|} \int_\Omega v_j D^\alpha \varphi \, dx = \\
        & = & (-1)^\alpha \int_\Omega v  D^\alpha \varphi \, dx.
\end{eqnarray*}
\hfill $\Box$

\bigskip

An alternative definition of Sobolev spaces were to take the closure
of smooth functions in the domain, i.e.,
$$
\widetilde W_p^k := \overline{ \{ C^\infty (\Omega) : \|.\|_{W_p^k} \leq \infty \} }^{\|.\|_{W_p^k}}.
$$
A third one is to take continuously differentiable functions up to the boundary
$$
\widehat W_p^k := \overline{ C^\infty (\overline{\Omega}) }^{\|.\|_{W_p^k}}.
$$

Under moderate restrictions, these definitions lead to the same spaces:
\begin{theorem} Let $1 \leq p < \infty$. Then $\widetilde W_p^k  = W_p^k$.
\end{theorem}

\begin{definition} The domain $\Omega$ has a {\bf Lipschitz boundary}, $\partial \Omega$, if there exists a collection of open sets $O_i$, a positive parameter $\varepsilon$, an integer $N$ and a finite number $L$, such that for all $x \in \partial \Omega$ the ball of radius $\varepsilon$ centered at $x$ is contained in some $O_i$, no more than $N$ of the sets $O_i$ intersect non-trivially, and each 
part of the boundary $O_i \cap \Omega$ is a graph of a Lipschitz function $\varphi_i : {\mathbb R}^{d-1} \rightarrow {\mathbb R}$ with Lipschitz norm bounded by $L$.
\end{definition}

\begin{theorem} Assume that $\Omega$ has a Lipschitz boundary, and let $1 \leq p < \infty$. Then  $\widehat W_p^k  = W_p^k$.
\end{theorem}


The case $W_2^k$ is special, it is a Hilbert space. We denote it by
$$
H^k(\Omega) := W_2^k(\Omega).
$$
The inner product is 
$$
(u,v)_{H^k} := \sum_{|\alpha| \leq k} (D^\alpha u, D^\alpha v)_{L_2}
$$
In the following, we will prove most theorems for the Hilbert spaces $H^k$,
and state the general results for $W_p^k$.












\section{Trace theorems and their applications}
We consider boundary values of functions in Sobolev spaces. Clearly,
this is not well defined for $H^0 = L_2$. But, as we will see, in
$H^1$ and higher order Sobolev spaces, it makes sense to talk about $u
|_{\partial \Omega}$. The definition of traces is essential to
formulate boundary conditions of PDEs in weak form.


We start in one dimension. Let $u \in C^1 ([0,h])$ with some $h > 0$. Then,
we can bound 
\begin{eqnarray*}
u(0) & = & \left(1 - \frac{x}{h} \right) u(x) |_{x=0} = 
        - \int_0^h \left\{ \left(1-\frac{x}{h} \right) u(x) \right\}^\prime \, dx \\
        & = & -\int_0^h  \frac{-1}{h} u(x) + \left(1-\frac{x}{h} \right) u^\prime(x)  \, dx \\
        & \leq & \left\| \frac{1}{h} \right\|_{L_2} \| u \|_{L_2} + 
                \left\| 1-\frac{x}{h} \right\|_{L_2} \| u^\prime \|_{L_2} \\
        & \eqc & h^{-1/2} \| u \|_{L_2(0,h)} + h^{1/2} \| u^\prime \|_{L_2(0,h)}.
\end{eqnarray*}
This estimate includes the scaling with the interval length $h$. 
If we are not interested in the scaling, we apply Cauchy-Schwarz in ${\mathbb R}^2$,
and combine the $L_2$ norm and the $H^1$ semi-norm $\|u^\prime\|_{L_2}$ to the full $H^1$ norm and obtain
$$
|u(0)| \leq \sqrt{h^{-1/2} + h^{1/2} } \sqrt{ \| u \|_{L_2}^2 + \| u^\prime \|_{L_2}^2 } = c \, \| u \|_{H^1}.
$$
Next, we extend the trace operator to the whole Sobolev space $H^1$:
\begin{theorem} There is a well defined and continuous trace operator
$$
\optr : H^1((0,h)) \rightarrow {\mathbb R}
$$
whose restriction to $C^1([0,h])$ coincides with
$$
u \rightarrow u(0).
$$
\end{theorem}
{\em Proof:} Use that $C^1([0,h])$ is dense in $H^1(0,h)$. Take a sequence
$u_j$ in $C^1([0,h])$ converging to $u$ in $H^1$-norm. The values 
$u_j(0)$ are Cauchy, and thus converge to an $u_0$. The limit is independent 
of the choice of the sequence $u_j$. This allows to define $\optr u := u_0$.
\hfill $\Box$

\medskip

Now, we extend this 1D result to domains in more dimensions. Let
$\Omega$ be bounded, $\partial \Omega$ be Lipschitz, and consists of
$M$ pieces $\Gamma_i$ of smoothness $C^1$. 

We can construct the following covering of a neighbourhood of 
$\partial \Omega$ in $\Omega$: Let $Q = (0,1)^2$. For $1 \leq i \leq M$, 
let $s_i \in C^1 (Q, \Omega)$ be invertible and such that $\left\| s_i^\prime \right\|_{L_\infty} \leq c$,
 $\| (s_i^\prime)^{-1} \|_{L_\infty} \leq c$, and $\operatorname{det} s_i^\prime > 0$. The domains $S_i := s_i(Q)$ are
such that $s_i( (0,1) \times \{ 0 \} ) = \Gamma_i$, and the parameterizations
match on $s_i( \{ 0,1 \} \times (0,1) )$.

\begin{theorem}
There exists a well defined and continuous operator
$$
\optr : H^1 (\Omega) \rightarrow L_2(\partial \Omega)
$$
which coincides with $u|_{\partial \Omega}$ for $u \in C^1(\overline{\Omega})$.
\end{theorem}
\noindent
{\em Proof:} Again, we prove that 
$$
\optr : C^1(\overline{\Omega}) \rightarrow L_2(\partial \Omega)
        : u \rightarrow u |_{\partial \Omega}
$$
is a bounded operator w.r.t. the norms $\|.\|_{H^1}$ and $L_2$, and conclude by
density.
We use the partitioning of $\partial \Omega$ into the pieces $\Gamma_i$, and
transform to the simple square domain $Q = (0,1)^2$. 
Define the functions $u_i$ on $Q = (0,1)^2$ as
$$
\tilde u_i (\tilde x) = u(s_i(\tilde x))
$$
We transfer the $L_2$ norm to the simple domain:
\begin{eqnarray*}
\| \optr u \|_{L_2(\partial \Omega)}^2 & = & 
%       \sum_{i=1}^M \| u \|_{L_2(\Gamma_i)}^2 =
        \sum_{i=1}^M \int_{\Gamma_i} u(x)^2 \, dx \\
        & = & 
        \sum_{i=1}^M \int_0^1 u(s_i(\xi,0))^2 \left| \frac{\partial s_i}{\partial \xi} (\xi, 0) \right| \, d \xi \\
        & \leq & c \sum_{i=1}^M \int_0^1 \tilde u_i(\xi,0)^2  \, d \xi  
\end{eqnarray*}
To transform the $H^1$-norm, 
we differentiate with respect to $\tilde x$ by applying the chain rule
$$
\frac{d \tilde u_i}{d \tilde x} (\tilde x) = \frac{d u}{d x} (s_i(\tilde x))
 \frac{d s_i}{d \tilde x} (\tilde x).
$$
Solving for $\frac{d u }{d x}$ is
$$
\frac{d u }{dx} (s_i(\tilde x)) = \frac{ d \tilde u_i}{d \tilde x}(\tilde x)
\left( \frac{ds }{d \tilde x} \right)^{-1} (\tilde x)
% \nabla_x u(s_i(\xi,\eta)) = \nabla_{(\xi,\eta)} \tilde u_i (\xi,\eta)\, (\nabla s)^{-1}.
$$
The bounds onto $s^\prime$ and $(s^\prime)^{-1}$ imply that
$$
c^{-1} \, |\nabla_x u | \leq | \nabla_{\tilde x} \tilde u | \leq c \, | \nabla_x u |
$$
We start from the right hand side of the stated estimate:
\begin{eqnarray*}
\| u \|_{H^1(\Omega)}^2 & \geq &
 \sum_{i=1}^M \int_{S_i} | \nabla_x u |^2 \, dx \\
 & = & \sum_{i=1}^M \int_Q \left| \nabla_x u (s_i (\tilde x)) \right|^2    \det ( s^\prime ) \, d \tilde x \\
 & \geq & c \,  \sum_{i=1}^M \int_Q \left| \nabla_{\tilde x} \tilde u (\tilde x) \right| ^2  \, d \tilde x
\end{eqnarray*}
We got a lower bound for $\det (s^\prime) = (\det (s^\prime)^{-1})^{-1}$ from the upper bound for $(s^\prime)^{-1}$.

It remains to prove the trace estimate on $Q$. Here, we apply the previous
one dimensional result 
$$
| u(\xi, 0)|^2 \leq c \int_0^1 \left\{ u(\xi,\eta)^2 + \left(\frac{\partial u(\xi,\eta)}{\partial \eta}\right)^2 \right\} \, d \eta
\qquad \forall \, \xi \in (0,1)
$$
The result follows from integrating over $\xi$
\begin{eqnarray*}
\int_0^1 |  u(\xi, 0)|^2 \, d\xi & \leq & 
 c \, \int_0^1  \int_0^1  \left\{ u(\xi,\eta)^2 + \left(\frac{\partial u(\xi,\eta)}{\partial \eta}\right)^2 \right\} \, d \eta \, d \xi \\
        & \leq & c \, \| u \|_{H^1(Q)}^2.
\end{eqnarray*}
\hfill $\Box$


\bigskip

Considering the trace operator from $H^1(\Omega)$ to $L_2(\partial \Omega)$ is
not sharp with respect to the norms. We will improve the embedding later.

\bigskip

By means of the trace operator we can define the sub-space
$$
H_0^1(\Omega) = \{ u \in H^1(\Omega) : \optr \, u = 0 \}
$$
It is a true sub-space, since $u = 1$ does belong to $H^1$, but not to $H_0^1$.
It is a closed sub-space, since it is the kernel of a continuous operator.


\bigskip

By means of the trace inequality, one verifies that the linear functional
$$
g(v) := \int_{\Gamma_N} g \, \optr \, v \, dx
$$
is bounded on $H^1$. 


\subsubsection{Integration by parts}
The definition of the trace allows us to perform integration by
parts in $H^1$:
$$
\int_{\Omega} \nabla u \varphi \, dx = -\int_\Omega u \opdiv \, \varphi \, dx 
        + \int_{\partial \Omega} \optr u \, \varphi \cdot n \, dx 
\qquad \forall \, \varphi \in [C^1(\overline{\Omega})]^2
$$
The definition of the weak derivative (e.g. the weak gradient) looks similar.
It allows only test functions $\varphi$ with compact support in $\Omega$, i.e.,
having zero boundary values. Only by choosing a normed space, for which the 
trace operator is well defined, we can state and prove integration by parts.
Again, the short proof is based on the density of $C^1(\overline \Omega)$ 
in $H^1$.

\subsubsection{Sobolev spaces over sub-domains}
Let $\Omega$ consist of $M$ Lipschitz-continuous sub-domains $\Omega_i$ 
such that
\begin{itemize}
\item $\overline \Omega = \cup_{i=1}^M \overline \Omega_i$
\item $\Omega_i \cap \Omega_j = \emptyset \quad \mbox{ if } i \neq j$
\end{itemize}
The interfaces are $\gamma_{ij} = \overline \Omega_i \cap \overline \Omega_j$.
The outer normal vector of $\Omega_i$ is $n_i$.

\begin{theorem} \label{theo_subdomainh1}
Let $u \in L_2(\Omega)$ such that
\begin{itemize}
\item
$u_i := u|_{\Omega_i}$ is in $H^1(\Omega_i)$, and $g_i = \nabla u_i$ is its
weak gradient
\item
the traces on common interfaces coincide:
$$
\optr_{\gamma_{ij}} u_i = 
\optr_{\gamma_{ij}} u_j 
$$
\end{itemize}
Then $u$ belongs to $H^1(\Omega)$. Its weak gradient $g = \nabla u$ fulfills
$g|_{\Omega_i} = g_i$.
\end{theorem}
{\em Proof:} We have to verify that $g \in L_2(\Omega)^d$, defined by $g |_{\Omega_i} = g_i$,
is the weak gradient of $u$, i.e.,
$$
\int_\Omega g \cdot \varphi \, dx = - \int_\Omega u \, \opdiv \varphi \, dx
\qquad \forall \, \varphi \in [C_0^\infty(\Omega)]^d
$$
We are using Green's formula on the sub-domains
\begin{eqnarray*}
\int_\Omega g \cdot \varphi \, dx & = & 
        \sum_{i=1}^M \int_{\Omega_i} g_i \cdot \varphi \, dx =
        \sum_{i=1}^M \int_{\Omega_i} \nabla u_i \cdot \varphi \, dx \\
        & = & 
        \sum_{i=1}^M -\int_{\Omega_i} u_i  \opdiv \varphi \, dx 
           + \int_{\partial \Omega_i} \optr u_i \, \varphi \cdot n_i \, ds \\
        & = & -\int_\Omega u \opdiv \varphi \, dx 
        + \sum_{\gamma_{ij}} \int_{\gamma_{ij}} 
        \left\{ \optr_{\gamma_{ij}} u_i \, \varphi \cdot n_i +
                \optr_{\gamma_{ij}} u_j \, \varphi \cdot n_j
        \right\} \, ds \\
        & = & -\int_\Omega u \, \opdiv \varphi \, dx
\end{eqnarray*}
We have used that $\varphi = 0$ on $\partial \Omega$, and $n_i = -n_j$ on $\gamma_{ij}$.
\hfill $\Box$

\bigskip

Applications of this theorem are (conforming nodal) finite element
spaces. The partitioning $\Omega_i$ is the mesh. On each sub-domain,
i.e., on each element $T$, the functions are polynomials and thus in
$H^1(T)$. The finite element functions are constructed to be
continuous, i.e., the traces match on the interfaces. Thus, the finite
element space is a sub-space of $H^1$.

\subsubsection{Extension operators}
Some estimates are elementary to verify on simple domains such as squares $Q$. 
One technique to transfer these results to general domains is to extend 
a function $u \in H^1(\Omega)$ onto a larger square $Q$, apply the result 
for the  square, and restrict the result onto the general domain $\Omega$.
This is now the motivation to study extension operators.

We construct a non-overlapping covering $\{ S_i \}$ of a neighbourhood 
of $\partial \Omega$ on both sides. 
Let $\partial \Omega = \cup \Gamma_i$ consist of smooth parts.
Let $s : (0,1) \times (-1,1) \rightarrow S_i : (\xi, \eta) \rightarrow x$ be
an invertible function such that 
\begin{eqnarray*}
s_i ( (0,1) \times (0,1) ) & = & S_i \cap \Omega \\
s_i ( (0,1) \times \{ 0 \} ) & = & \Gamma_i \\
s_i ( (0,1) \times (-1,0) ) & = & S_i \setminus \overline \Omega 
\end{eqnarray*}
Assume that $\| \frac{ds_i}{dx} \|_{L_\infty}$ and $\| \left( \frac{ds_i}{dx} \right)^{-1} \|_{L_\infty}$ are bounded. 

This defines an invertible mapping $x \rightarrow \hat x (x)$ from the inside 
to the outside by
$$
\hat x(x) = s_i (\xi(x), -\eta(x)).
$$
The mapping preserve the boundary $\Gamma_i$.
The transformations $s_i$ should be such that $x \rightarrow \hat x$ is
consistent at the interfaces between $S_i$ and $S_j$.

With the flipping operator $f : (\xi, \eta) \rightarrow (\xi, -\eta)$, the
mapping is the composite $\hat x(x) = s_i (f(s_i^{-1}))$. From that, we
obtain the bound
$$
\left\| \frac{d \hat x}{d x } \right\| \leq
         \left\| \frac{ds}{dx} \right\|
         \left\| \left( \frac{ds}{dx} \right)^{-1} \right\|.
$$
Define the domain $\widetilde \Omega = \Omega \cup S_1 \cup \ldots \cup S_M$.

We define the extension operator by
\begin{equation} \label{extension}
\begin{array}{rcll}
(E u) (\hat x) & = & u(x) & \qquad \forall \, x \in \cup S_i \\
(Eu) (x) & = & u(x) & \qquad \forall \, x \in \Omega
\end{array}
\end{equation}

\begin{theorem} The extension operator $E : H^1(\Omega) \rightarrow H^1(\widetilde \Omega)$ is well defined and bounded with respect to the norms
$$
\| E u \|_{L_2(\widetilde \Omega)} \leq c \, \| u \|_{L_2(\Omega)}
$$
and
$$
\| \nabla E u \|_{L_2(\widetilde \Omega)} \leq c \, \| \nabla u \|_{L_2(\Omega)}
$$
\end{theorem}
{\em Proof:} Let $u \in C^1(\overline \Omega)$. First, we prove the estimates
for the individual pieces $S_i$:
$$
\int_{S_i \setminus \Omega} E u (\hat x)^2 \, d \hat x =
\int_{S_i \cap \Omega} u (x)^2 \, \det \left( \frac{d \hat x}{d x} \right) d x \leq
c \| u \|_{L_2(S_i \cap \Omega)}^2 
$$
For the derivatives we use
$$
\frac{d E u(\hat x)}{d\hat x} = \frac{d u(x(\hat x))}{d \hat x} =
        \frac{d u}{dx} \frac{dx}{d \hat x}.
$$
Since $\frac{d x}{d \hat x}$ and $(\frac{d x}{d \hat x})^{-1} = \frac{d \hat x}{dx}$ are bounded, one obtains
$$
| \nabla_{\hat x} E u (\hat x) | \eqc | \nabla_x u(x) |,
$$
and 
$$
\int_{S_i \setminus \Omega} | \nabla_{\hat x} E u |^2 \, d \hat x \leq 
        c \int_{S_i \cap \Omega} | \nabla u |^2 \, dx
$$
These estimates prove that $E$ is a bounded operator into $H^1$ on the
sub-domains $S_i \setminus \Omega$. The construction 
was such that for $u \in C^1(\overline \Omega)$, the extension $E u$ is 
continuous across $\partial \Omega$, and also across the individual $S_i$.
By Theorem~\ref{theo_subdomainh1}, $E u $ belongs
to $H^1 (\widetilde \Omega)$, and
$$
\| \nabla E u \|_{L_2(\widetilde \Omega)}^2 = 
\| \nabla u \|_{\Omega}^2 + \sum_{i=1}^M \| \nabla u \|_{S_i \setminus \Omega}^2 \leq c \| \nabla u \|_{L_2(\Omega)}^2,
$$
By density, we get the result for $H^1(\Omega)$. Let $u_j  \in C^1(\overline \Omega) \rightarrow  u$, than $u_j$ is Cauchy, $E u_j$ is Cauchy in $H^1(\widetilde \Omega)$, and thus converges to $u \in H^1(\widetilde \Omega)$.

\bigskip

The extension of functions from $H_0^1(\Omega)$ onto larger domains is
trivial: Extension by $0$ is a bounded operator.
One can extend functions from $H^1(\Omega)$ into $H_0^1(\widetilde \Omega)$,
and further, to an arbitrary domain by extension by $0$.

For $\hat x = s_i(\xi, -\eta)$, $\xi, \eta \in (0,1)^2$, define the extension
$$
E_0 u (\hat x) = (1-\eta) \, u(x) 
$$
This extension vanishes at $\partial \widetilde \Omega$

\begin{theorem}
The extension $E_0$ is an extension from $H^1(\Omega)$ to $H_0^1(\widetilde \Omega)$. It is bounded w.r.t. 
$$
\| E_0 u \|_{H^1(\widetilde \Omega)} \leq c \| u \|_{H^1(\Omega)}
$$
\end{theorem}
\noindent
{\em Proof:} Exercises

In this case, it is not possible to bound the gradient term only by gradients.
To see this, take the constant function on $\Omega$. The gradient vanishes,
but the extension is not constant.


\subsection{The trace space $H^{1/2}$ }
\label{sec_traceh1}
%
The trace operator is continuous from $H^1(\Omega)$ into $L_2(\partial \Omega)$. But, not every $g \in L_2(\partial \Omega)$ is a trace of some $u \in H^1(\Omega)$. We will motivate why the trace space is the fractional order Sobolev 
space $H^{1/2}(\partial \Omega)$.

We introduce a stronger space, such that the trace operator is still
continuous, and onto. 
Let $V = H^1(\Omega)$, and define the trace space as the range of the 
trace operator
$$
W = \{ \optr \, u : u \in H^1(\Omega) \}
$$
with the norm
\begin{equation}
\label{equ_tracenorm}
\| \optr u \|_W = \inf_{v \in V \atop \optr \, u = \optr \, v } \| v \|_V.
\end{equation}
This is indeed a norm on $W$. The trace operator is continuous from $V \rightarrow W$ with norm $1$. 

\begin{lemma} The space $(W, \|.\|_W)$ is a Banach space. For all $g \in W$
there exists an $u \in V$ such that $\optr \, u = g$ and $\| u \|_V = \| g \|_W$
\end{lemma}
\noindent {\em Proof:}
The kernel space $V_0 := \{ v : \optr \, v = 0 \}$ is a closed sub-space of 
$V$. If $\optr \, u = \optr \, v$, then $z := u - v \in V_0$. We can rewrite
$$
\| \optr \, u \|_W = \inf_{z \in V_0} \| u - z \|_V = \| u - P_{V_0} u \|_V
\qquad  \forall \, u \in V
$$
Now, let $g_n = \optr \, u_n \in W$ be a Cauchy sequence. This does not
imply that $u_n$ is Cauchy, but $P_{V_0^\bot} u_n$ is 
Cauchy in $V$:
$$
\| P_{V_0^\bot} (u_n - u_m) \|_V = \| \optr \, (u_n - u_m) \|_W.
$$
The $P_{V_0^\bot} u_n$ converge to some $u \in V_0^\bot$, and $g_n$ converge
to $g := \optr \, u$. 
\hfill $\Box$


The minimizer in (\ref{equ_tracenorm}) fulfills
$$
\optr \, u = g \qquad \mbox{and} \qquad (u,v)_V = 0 \qquad \forall \, v \in V_0.
$$
This means that $u$ is the solution of the weak form of the
Dirichlet problem 
$$
\begin{array}{rcll}
-\Delta u + u & = & 0 \qquad & \mbox{ in }  \Omega \\
u & = & g \qquad & \mbox{ on } \partial \Omega.
\end{array}
$$


\bigskip

To give an explicit characterization of the norm $\|.\|_W$, we introduce
{\bf Hilbert space interpolation}:

Let $V_1 \subset V_0$ be two Hilbert spaces, such that $V_1$ is dense 
in $V_0$, and the embedding operator $id : V_1 \rightarrow V_0$
is compact. We can pose the eigen-value problem: Find $z \in V_1$, $\lambda \in {\mathbb R}$ such that
$$
(z, v)_{V_1} = \lambda (z, v)_{V_0} \qquad \forall \, v \in V_1.
$$
There exists a sequence of eigen-pairs $(z_k, \lambda_k)$ such that $\lambda_k
\rightarrow \infty$. The $z_k$ form an orthonormal basis in $V_0$, and
an orthogonal basis in $V_1$. 

\bigskip

The converse is also true. If $z_k$ is a basis for $V_0$, and the eigenvalues
$\lambda_k \rightarrow \infty$, then the embedding $V_1 \subset V_0$ is compact.

\bigskip

Given $u \in V_0$, it can be expanded in the orthonormal eigen-vector basis:
$$
u = \sum_{k=0}^\infty u_k z_k \qquad \mbox{with} \qquad u_k = (u, z_k)_{V_0}
$$
The $\|.\|_{V_0}$ - norm of $u$ is 
$$
\| u \|_{V_0}^2 = (\sum_k u_k z_k, \sum_l u_l z_l)_{V_0} = 
        \sum_{k,l} u_k u_l (z_k,z_l)_{V_0} = \sum_k u_k^2.
$$
If $u \in V_1$, then 
$$
\| u \|_{V_1}^2 = (\sum_k u_k z_k, \sum_l u_l z_l)_{V_1} = 
        \sum_{k,l} u_k u_l (z_k,z_l)_{V_1} = 
        \sum_{k,l} u_k u_l \lambda_k (z_k,z_l)_{V_0} = 
        \sum_k u_k^2 \lambda_k
$$
The sub-space space $V_1$ consists of all $u = \sum u_k z_k$ such 
that $\sum_k \lambda_k u_k^2$ is finite. This suggests the definition
of the interpolation norm
$$
\| u \|_{V_s}^2 = \sum_k (u, z_k)_{V_0}^2 \lambda_k^s,
$$
and the interpolation space $V_s = [V_0, V_1]_s$ as
$$
V_s = \{ u \in V_0 : \| u \|_{V_s} < \infty \}.
$$
We have been fast with using infinite sums. To make everything precise,
one first works with finite dimensional sub-spaces $\{ u : \exists n \in {\mathbb N} \mbox{ and } u = \sum_{k=1}^n u_k z_k \}$, and takes the closure.


\bigskip

In our case, we apply Hilbert space interpolation to $H^1(0,1) \subset L_2(0,1)$. The eigen-value problem is to find $z_k \in H^1$ and $\lambda_k \in {\mathbb R}$
such that
$$
(z_k, v)_{L_2} + (z_k^\prime, v^\prime)_{L_2} = \lambda_k \, 
        (z_k, v)_{L_2} \qquad \forall \, v \in H^1
$$
By definition of the weak derivative, there holds $(z_k^\prime)^\prime = (1-\lambda_k) z_k$, i.e., $z^k \in H^2$. Since $H^2 \subset C^0$, there holds also
$z \in C^2$, and a weak solution is also a solution of the strong form
\begin{equation}
\begin{array}{rcll}
z_k - z_k^{\prime \prime} & = & \lambda_k z_k \qquad & \mbox{ on } (0,1) \\
z_k^\prime(0) = z_k^\prime(1) & = & 0
\end{array}
\end{equation}
All solutions, normalized to $\| z_k \|_{L_2} = 1$, are
$$
z_0 = 1 \qquad \lambda_0 = 1
$$
and, for $k \in {\mathbb N}$,
$$
z_k(x) = \sqrt{2} \cos (k \pi x) \qquad \lambda_k = 1 + k^2 \pi^2.
$$
Indeed, expanding $u \in L_2$ in the $\cos$-basis $u = u_0 + \sum_{k=1}^\infty
u_k \sqrt2 \cos (k \pi x)$, one has
$$
\| u \|_{L_2}^2 = \sum_{k=0}^\infty (u, z_k)_{L_2}^2
$$
and
$$
\| u \|_{H^1}^2 = \sum_{k=0}^\infty (1+k^2 \pi^2) (u, z_k)_{L_2}^2
$$
Differentiation adds a factor $k \pi$.
Hilbert space interpolation allows to define the fractional order Sobolev
norm ($s \in (0,1)$)
$$
\| u \|_{H^s(0,1)}^2 = \sum_{k=0}^\infty (1+k^2 \pi^2)^s (u,z_k)_{L_2}^2
$$




\bigskip
We consider the trace $\optr|_E$ of $H^1((0,1)^2)$ onto one edge $E = (0,1) \times \{ 0 \}$. For $g \in W_E := \optr H^1((0,1)^2)$, the norm $\| g \|_W$ is 
defined by 
$$
\| g \|_W = \| u_g \|_{H^1}.
$$
Here, $u_g$ solves the Dirichlet problem $u_g|_E = g$, and $(u_g,v)_{H^1} = 0 \; \forall \, v \in H^1$ such that $\optr_E v = 0$.

Since $W \subset L_2(E)$, we can expand $g$ in the $L_2$-orthonormal cosine basis $z_k$
$$
g(x) = \sum g_n z_k(x)
$$
The Dirichlet problems for the $z_k$,
$$
\begin{array}{rcll}
-\Delta u_k + u_k & = & 0  \qquad & \mbox{in } \Omega \\
 u_k & = & z_k \qquad & \mbox{on } E \\
\frac{\partial u_k}{\partial n} & = & 0 \qquad & \mbox{on } \partial \Omega \setminus E,
\end{array}
$$
have the explicit solution
$$
u_0(x,y) = 1
$$
and
$$
u_k(x,y) = \sqrt 2 \cos (k \pi x) \frac{e^{k \pi (1-y)} + e^{-k \pi (1-y)}}{e^{k \pi} + e^{-k \pi}}.
$$
The asymptotic is
$$
\| u_k \|_{L_2}^2 \eqc (k+1)^{-1}
$$
and
$$
\| \nabla u_k \|_{L_2}^2 \eqc k 
$$
Furthermore, the $u_k$ are orthogonal in $(.,.)_{H^1}$. Thus $u_g = \sum_n g_n u_k$ has the norm
$$
\| u_g \|_{H^1}^2 = \sum g_n^2 \| u_k \|_{H^1}^2 \eqc \sum g_n^2 (1+k).
$$
This norm is equivalent to $H^{1/2}(E)$. 


We have proven that the trace space onto one edge is the interpolation 
space $H^{1/2}(E)$. This is also true for general domains (Lipschitz, with piecewise smooth boundary). 










\section{Equivalent norms on $H^1$ and on sub-spaces}

The intention is to formulate $2^{nd}$ order variational problems in the 
Hilbert space $H^1$. We want to apply the Lax-Milgram theory for continuous
and coercive bilinear forms $A(.,.)$. We present techniques to prove
coercivity.

The idea is the following. In the norm
$$
\| v \|_{H^1}^2 = \| v \|_{L_2}^2 + \| \nabla v \|_{L_2}^2,
$$
%
the $\| \nabla \cdot \|_{L_2}$-semi-norm is the dominating part up to
the constant functions. The $L_2$ norm is necessary to obtain a
norm. We want to replace the $L_2$ norm by some different term (e.g.,
the $L_2$-norm on a part of $\Omega$, or the $L_2$-norm on $\partial
\Omega$), and want to obtain an equivalent norm.


We formulate an abstract theorem relating a norm $\|.\|_V$ to a semi-norm
$\|.\|_A$. An equivalent theorem was proven by Tartar.
\begin{theorem} [Tartar] \label{theo_tartar}
Let $(V, (.,.)_V)$ and $(W, (.,.)_W)$ be Hilbert spaces, such that
the embedding $id : V \rightarrow W$ is compact.
Let $A(.,.)$ be a non-negative, symmetric and $V$-continuous bilinear form 
with kernel $V_0 = \{ v : A(v,v) = 0 \}$. Assume that
\begin{equation}
\label{equ_tartar_cond}
\| v \|_V^2 \eqc \| v \|_W^2 + \| v \|_A^2 \qquad \forall \, v \in V
\end{equation}
Then there holds
\begin{enumerate}
\item
The kernel $V_0$ is finite dimensional. On the factor space $V/V_0$, 
$A(.,.)$ is an equivalent norm to the quotient norm
\begin{equation} \label{equ_factor}
\| u \|_A \eqc \inf_{v \in V_0} \| u - v \|_V \qquad \forall \, u \in V
\end{equation}
\item
Let $B(.,.)$ be a continuous, non-negative, symmetric bilinear form 
on $V$ such that $A(.,.) + B(.,.)$ is an inner product. Then there
holds
$$
\| v \|_V^2 \eqc \| v \|_A^2 + \| v \|_B^2 \qquad \forall \, v \in V
$$
\item
Let $V_1 \subset V$ be a closed sub-space such that $V_0 \cap V_1 = \{ 0 \}$.
Then there holds
$$
\| v \|_V \eqc \| v \|_A \qquad \forall \, v \in V_1
$$
\end{enumerate}
\end{theorem}
\noindent
{\em Proof:} 1. Assume that $V_0$ is not finite dimensional. Then 
there exists an $(.,.)_V$-orthonormal sequence $u_k \in V_0$. Since
the embedding $id : V \rightarrow W$ is compact, it has a sub-sequence
converging in $\|. \|_W$. But, since
$$
2 = \| u_k - u_l \|_V^2 \eqc \| u_k - u_l \|_W^2 + \| u_k - u_l \|_A
        = \| u_k - u_l \|_W^2
$$ 
for $k \neq l$, $u_k$ is not Cauchy in $W$. This is a contradiction to
an infinite dimensional kernel space $V_0$.
We prove the equivalence (\ref{equ_factor}). To bound the left hand side
by the right hand side, we use that $V_0 = \operatorname {ker} A$, and
norm equivalence (\ref{equ_tartar_cond}):
$$
\| u \|_A = \inf_{v \in V_0} \| u - v \|_A \leq \inf_{v \in V_0} \| u - v \|_V 
$$
The quotient norm is equal to $\| P_{V_0^\bot} u \|$. We have to prove that 
$\| P_{V_0^\bot} u \|_V \leq \| P_{V_0^\bot} u \|_A$ for all $u \in V$.
This follows after proving $\| u \|_V \leq \| u \|_A$ for all $u \in V_0^\bot$.
Assume that this is not true. I.e., there exists a $V$-orthogonal sequence 
$(u_k)$ such that $\| u_k \|_A \leq k^{-1} \| u_k \|_V$. Extract a sub-sequence
converging in $\| . \|_W$, and call it $u_k$ again. From the norm
equivalence (\ref{equ_tartar_cond}) there follows
$$
2 = \| u_k - u_l \|_V^2 \leqc \| u_k - u_l \|_W + \| u_k - u_l \|_A \rightarrow 0
$$
2. On $V_0$, $\| . \|_B$ is a norm. Since $V_0$ is finite dimensional, it
is equivalent to $\|.\|_V$, say with bounds
$$
c_1 \, \| v \|_{V}^2 \leq \| v \|_B^2 \leq c_2 \, \| v \|_V^2 \qquad \forall \, v \in V_0
$$
From 1. we know that
$$
c_3 \, \| v \|_V^2 \leq \| v \|_A^2 \leq c_4 \, \| v \|_V^2 \qquad \forall \, v \in V_0^\bot.
$$
Now, we bound
\begin{eqnarray*}
\| u \|_V^2 & = & \| P_{V_0} u \|_V^2 + \| P_{V_0^\bot} u \|_V^2 \\
        & \leq & \frac{1}{c_1} \| \underbrace{P_{V_0} u }_{u - P_{V_0^\bot} u } \|_B^2 + \| P_{V_0^\bot} \|_V^2 \\
        & \leq & \frac{2}{c_1} \left( \| u \|_B^2 + c_2 \| P_{V_0^\bot} u \|_V^2 \right) + \| P_{V_0^\bot} u \|_V^2 \\
        & = & \frac{2}{c_1} \| u \|_B^2 + \frac{1}{c_2} \left(1 + \frac{2 c_2}{c_1} \right) \| P_{V_0^\bot} u \|_A^2 \\
        & \leqc & \| u \|_B^2 + \| u \|_A^2
\end{eqnarray*}

3. Define $B(u,v) = (P_{V_1}^\bot u, P_{V_1}^\bot u)_V$. Then $A(.,.)+B(.,.)$
is an inner product: $A(u,u)+B(u,u) = 0$ implies that $u \in V_0$ and $u \in V_1$, thus $u = \{ 0 \}$. From 2. there follows that $A(.,.)+B(.,.)$ is
equivalent to $(.,.)_V$. The result follows from reducing the equivalence 
to $V_1$.

\hfill $\Box$

We want to apply Tartar's theorem to the case $V = H^1$, $W = L_2$,
and $\|v \|_A = \| \nabla v \|_{L_2}$.
The theorem requires that the embedding $id : H^1 \rightarrow L_2$ is
compact. This is indeed true for bounded domains $\Omega$:

\begin{theorem} The embedding of $H^k \rightarrow H^l$ for $k > l$ is 
compact.
\end{theorem}
We sketch a proof for the embedding $H^1 \subset L_2$.
First, prove the compact embedding $H_0^1(Q) \rightarrow L_2(Q)$ for
a square $Q$, w.l.o.g. set $Q = (0,1)^2$. 
The eigen-value problem: Find $z \in H_0^1(Q)$ and $\lambda$ such that
$$
(z,v)_{L_2} + (\nabla z, \nabla v)_{L_2} = \lambda (u,v)_{L_2} \qquad 
\forall \, v \in H_0^1(Q)
$$
has eigen-vectors $z_{k,l} = sin (k \pi x) sin (l \pi y)$, and eigen-values
$1+k^2 \pi^2 + l^2 \pi^2 \rightarrow \infty$. The eigen-vectors are dense 
in $L_2$. Thus, the embedding is compact.

On a general domain $\Omega \subset Q$, we can extend $H^1(\Omega)$ 
into $H_0^1(Q)$, embed $H_0^1(Q)$ into $L_2(Q)$, and restrict $L_2(Q)$
onto $L_2(\Omega)$. This is the composite of two continuous and a compact
mapping, and thus is compact.
\hfill $\Box$

\bigskip
\noindent

The kernel $V_0$ of the semi-norm $\| \nabla v \|$ is the constant function.

\begin{theorem} [Friedrichs inequality]
Let $\Gamma_D \subset \partial \Omega$ be of positive measure $|\Gamma_D|$.
Let $V_D = \{ v \in H^1(\Omega) : \optr_{\Gamma_D} v = 0 \}$. Then
$$
\| v \|_{L_2} \leqc \| \nabla v \|_{L_2} \qquad \forall \, v \in V_D
$$
\end{theorem}
{\em Proof:} The intersection $V_0 \cap V_D$ is trivial $\{ 0 \}$. 
Thus, Theorem~\ref{theo_tartar}, 3. implies the equivalence  
$$
\| v \|_V^2 = \| v \|_{L_2}^2 + \| \nabla v \|_{L_2}^2 \eqc
\| \nabla v \|_{L_2}.
$$
\hfill $\Box$


\begin{theorem} [Poincar\'e inequality]
There holds
$$
\| v \|_{H^1(\Omega)}^2 \leqc \| \nabla v \|_{L_2}^2 + (\int_\Omega v \, dx)^2
$$
\end{theorem}
{\em Proof:} $B(u,v) := (\int_\Omega u \, dx) (\int_\Omega v \, dx)$ is
a continuous bilinear form on $H^1$, and $(\nabla u, \nabla v) + B(u,v)$ is
an inner product. Thus, Theorem~\ref{theo_tartar}, 2. implies the 
stated equivalence.
\hfill $\Box$

\begin{itemize}
\item
Let $\omega \subset \Omega$ have positive measure $| \omega|$ in ${\mathbb R}^d$.
Then
$$
\| u \|_{H^1(\Omega)}^2 \eqc
 \| \nabla v \|_{L_2(\Omega)}^2 + \| v \|_{L_2(\omega)},
$$
\item
Let $\gamma \subset \partial \Omega$ have positive measure $| \gamma|$ in ${\mathbb R}^{d-1}$. Then
$$
\| u \|_{H^1(\Omega)}^2 \eqc
\| \nabla v \|_{L_2(\Omega)}^2 + \| v \|_{L_2(\gamma)},
$$
\end{itemize}


\begin{theorem}[Bramble Hilbert lemma] \label{lemma_bh} 
Let $U$ be some Hilbert space, and
$L : H^k \rightarrow U$ be a continuous linear operator such that 
$L q = 0$ for polynomials $q \in P^{k-1}$. Then there holds
$$
\| L v \|_U \leq | v |_{H^k}.
$$
\end{theorem}
{\em Proof:} The embedding $H^k \rightarrow H^{k-1}$ is compact. 
The $V$-continuous, symmetric and non-negative bilinear form
$A(u,v) = \sum_{\alpha : | \alpha | = k} (\partial^\alpha u, \partial^\alpha v)$ has the kernel $P^{k-1}$. Decompose $\| u \|_{H^k}^2 = \| u \|_{H^{k-1}}^2 + A(u,u)$. By Theorem~\ref{theo_tartar}, 1, there holds
$$
\| u \|_A \eqc \inf_{v \in V_0} \| u - v \|_{H^k}
$$
The same holds for the bilinear-form
$$
A_2(u,v) := (L u, Lv)_U + A(u,v)
$$
Thus
$$
\| u \|_{A_2} \eqc  \inf_{v \in V_0} \| u - v \|_{H^k} \qquad \forall \, u \in V
$$
Equalizing both  implies that
$$
(L u, Lu)_U \leq \| u \|_{A_2}^2 \eqc \| u \|_A^2  \qquad \forall \, u \in V,
$$
i.e., the claim.

\bigskip


We will need point evaluation of functions in Sobolev spaces $H^s$. This is
possible, we $u \in H^s$ implies that $u$ is continuous. 
\begin{theorem}[Sobolev's embedding theorem] Let $\Omega \subset {\mathbb R}^d$ with Lipschitz boundary. If $u \in H^s$ with $s > d/2$, then $u \in L_\infty$
with
$$
\| u \|_{L_\infty} \leqc \| u \|_{H^s}
$$
There is a function in $C^0$ within the $L_\infty$ equivalence class.
\end{theorem}


\input{interpolation.tex}
